%% For double-blind review submission, w/o CCS and ACM Reference (max submission space)
\documentclass[acmsmall,review,anonymous]{acmart}\settopmatter{printfolios=true,printccs=false,printacmref=false}
%% For double-blind review submission, w/ CCS and ACM Reference
%\documentclass[acmsmall,review,anonymous]{acmart}\settopmatter{printfolios=true}
%% For single-blind review submission, w/o CCS and ACM Reference (max submission space)
%\documentclass[acmsmall,review]{acmart}\settopmatter{printfolios=true,printccs=false,printacmref=false}
%% For single-blind review submission, w/ CCS and ACM Reference
%\documentclass[acmsmall,review]{acmart}\settopmatter{printfolios=true}
%% For final camera-ready submission, w/ required CCS and ACM Reference
%\documentclass[acmsmall]{acmart}\settopmatter{}


%% Journal information
%% Supplied to authors by publisher for camera-ready submission;
%% use defaults for review submission.
\acmJournal{PACMPL}
\acmVolume{1}
\acmNumber{ICFP} % CONF = POPL or ICFP or OOPSLA
\acmArticle{1}
\acmYear{2018}
\acmMonth{1}
\acmDOI{} % \acmDOI{10.1145/nnnnnnn.nnnnnnn}
\startPage{1}

%% Copyright information
%% Supplied to authors (based on authors' rights management selection;
%% see authors.acm.org) by publisher for camera-ready submission;
%% use 'none' for review submission.
\setcopyright{none}
%\setcopyright{acmcopyright}
%\setcopyright{acmlicensed}
%\setcopyright{rightsretained}
%\copyrightyear{2018}           %% If different from \acmYear

%% Bibliography style
\bibliographystyle{ACM-Reference-Format}
%% Citation style
%% Note: author/year citations are required for papers published as an
%% issue of PACMPL.
\citestyle{acmauthoryear}   %% For author/year citations


%%%%%%%%%%%%%%%%%%%%%%%%%%%%%%%%%%%%%%%%%%%%%%%%%%%%%%%%%%%%%%%%%%%%%%
%% Note: Authors migrating a paper from PACMPL format to traditional
%% SIGPLAN proceedings format must update the '\documentclass' and
%% topmatter commands above; see 'acmart-sigplanproc-template.tex'.
%%%%%%%%%%%%%%%%%%%%%%%%%%%%%%%%%%%%%%%%%%%%%%%%%%%%%%%%%%%%%%%%%%%%%%


%% Some recommended packages.
\usepackage{booktabs}   %% For formal tables:
                        %% http://ctan.org/pkg/booktabs
\usepackage{subcaption} %% For complex figures with subfigures/subcaptions
                        %% http://ctan.org/pkg/subcaption

\begin{document}

%% Title information
\title[Short Title]{Pushdown Control-Flow Analysis via Refunctionalization}         %% [Short Title] is optional;
                                        %% when present, will be used in
                                        %% header instead of Full Title.
\titlenote{with title note}             %% \titlenote is optional;
                                        %% can be repeated if necessary;
                                        %% contents suppressed with 'anonymous'
\subtitle{Filling the Gap between Abstracting Abstract Machine and Abstracting Definitional Interpreter}        %% \subtitle is optional
\subtitlenote{with subtitle note}       %% \subtitlenote is optional;
                                        %% can be repeated if necessary;
                                        %% contents suppressed with 'anonymous'


%% Author information
%% Contents and number of authors suppressed with 'anonymous'.
%% Each author should be introduced by \author, followed by
%% \authornote (optional), \orcid (optional), \affiliation, and
%% \email.
%% An author may have multiple affiliations and/or emails; repeat the
%% appropriate command.
%% Many elements are not rendered, but should be provided for metadata
%% extraction tools.

%% Author with single affiliation.
\author{First1 Last1}
\authornote{with author1 note}          %% \authornote is optional;
                                        %% can be repeated if necessary
\orcid{nnnn-nnnn-nnnn-nnnn}             %% \orcid is optional
\affiliation{
  \position{Position1}
  \department{Department1}              %% \department is recommended
  \institution{Institution1}            %% \institution is required
  \streetaddress{Street1 Address1}
  \city{City1}
  \state{State1}
  \postcode{Post-Code1}
  \country{Country1}                    %% \country is recommended
}
\email{first1.last1@inst1.edu}          %% \email is recommended

%% Author with two affiliations and emails.
\author{First2 Last2}
\authornote{with author2 note}          %% \authornote is optional;
                                        %% can be repeated if necessary
\orcid{nnnn-nnnn-nnnn-nnnn}             %% \orcid is optional
\affiliation{
  \position{Position2a}
  \department{Department2a}             %% \department is recommended
  \institution{Institution2a}           %% \institution is required
  \streetaddress{Street2a Address2a}
  \city{City2a}
  \state{State2a}
  \postcode{Post-Code2a}
  \country{Country2a}                   %% \country is recommended
}
\email{first2.last2@inst2a.com}         %% \email is recommended
\affiliation{
  \position{Position2b}
  \department{Department2b}             %% \department is recommended
  \institution{Institution2b}           %% \institution is required
  \streetaddress{Street3b Address2b}
  \city{City2b}
  \state{State2b}
  \postcode{Post-Code2b}
  \country{Country2b}                   %% \country is recommended
}
\email{first2.last2@inst2b.org}         %% \email is recommended


%% Abstract
%% Note: \begin{abstract}...\end{abstract} environment must come
%% before \maketitle command
\begin{abstract}
  Abstracting abstract machines (AAM) is a systematic methodology for constructing abstract interpreter
  from a concrete small-step abstract machine. Recent progress applies the same idea on concrete definitional 
  interpreter and obtains a big-step abstracting definitional interpreter (ADI) written in monadic style.
  But the relation between small-step AAM and big-step ADI is not well-studied.
  
  % Mention Danvy's work
  In this paper, we show their correspondence and a series of transformations from a small-step 
  abstracting abstract machine to a big-step abstracting definitional interpreter. 
  The transformations include fusing, disentangling, 
  refunctionalizing and then monadification/un-CPS, each of them properly handle the collecting semantics and 
  the non-determinism of abstract interpretation. 
  The transformations also uncover several intermediary forms of abstract interpreter.
  Remarkably, we reveal how computatble precise call/return match in control-flow analysis is obtained by 
  refunctionalizing a small-step abstracting abstract machine with unbounded stack, 
  as well as explain how it been lost by defunctionalizing the higher-order continuations into 
  an algebraic data type.
  
  TODO: Shiver's formulation of CFA in denotational style.
  %Further, from another direction, we show Shiver's original implementation of control-flow analysis
  %in denotational style can be also transformed into natural semantics by closure conversion.
 
\end{abstract}

%% 2012 ACM Computing Classification System (CSS) concepts
%% Generate at 'http://dl.acm.org/ccs/ccs.cfm'.
\begin{CCSXML}
<ccs2012>
<concept>
<concept_id>10011007.10011006.10011008</concept_id>
<concept_desc>Software and its engineering~General programming languages</concept_desc>
<concept_significance>500</concept_significance>
</concept>
<concept>
<concept_id>10003456.10003457.10003521.10003525</concept_id>
<concept_desc>Social and professional topics~History of programming languages</concept_desc>
<concept_significance>300</concept_significance>
</concept>
</ccs2012>
\end{CCSXML}

\ccsdesc[500]{Software and its engineering~General programming languages}
\ccsdesc[300]{Social and professional topics~History of programming languages}
%% End of generated code


%% Keywords
%% comma separated list
\keywords{pushdown analysis, abstract interpretation, static analysis}  %% \keywords are mandatory in final camera-ready submission


%% \maketitle
%% Note: \maketitle command must come after title commands, author
%% commands, abstract environment, Computing Classification System
%% environment and commands, and keywords command.
\maketitle

\section{Introduction}

\subsection{Motivation}

\subsection{Contributions}

\begin{itemize}
  \item We fill the gap between small-step AAM and big-step ADI by developing a series of systematic transformations from small-step AAM to big-step ADI, and also show that the reversed direction is also possible.

  \item We present an abstract interpreter written in CPS obtained from refunctionalization of small-step AAM. This abstract interpreter naturally has pushdown control-flow property, and we show how precise call/return match is obtained by representing evaluation context as a continuation function, as well as how it is lost by representing continuation as a data type in the defining language. We also analyze its termination, soundness and complexity property.

  \item We show that the correspondence not only exists between concrete abstract machines and evaluator, and also exists for abstracting abstract machines and abstract evaluators. Galois connection?
\end{itemize}


\subsection{Outline}

\section{Background}

In this section, we first describe the syntax of A-Normal Form (ANF) $\lambda$-calculus and a CESK machine for executing it concretely. Then we review several recent works in control flow analysis: 1) Abstracting Abstract Machine as an abstract counterpart of CESK machine, 2) abstract interpreter written in a monadically-parameterized style, and 3) precise call/return match problem and existing solutions.

\subsection{A-Normal Form $\lambda$-Calculus}

\subsection{CESK Machine}

\subsection{Abstracting Abstract Machine}

\subsection{Monadic Abstract Interpreter}

\subsection{Call/Return Match in Control-Flow Analysis}

\subsubsection{AAM with Unbound Stack}

\subsubsection{Computable Solutions}
P4F ANF, not works on direct style,PDCFA,AAC

\section{FROM AAM TO ADI}

We start from the AAM with unbounded stack.

\subsection{Fusing}

\subsection{Disentangling}

\subsection{Refunctionalization}

allocating continuations of defined language on the heap of defining language.


%probably Store Widening \& Abstract Garbage Collection
%monadic
\subsection{Monadification: From Refunctionalized AAM to ADI}

For one direction, we can transform the refunctionalized AAM to abstracting definitional interpreter
by adding monads.

\subsection{Un-CPS: From Refunctionalized AAM to Direct Style}

In another direction, we can transform the refunctionalized AAM to direct style, but explictly use 
side effects such as assignment and mutation to handle collecting semantics and non-determinism.

\section{Refunctionalized AAM}

\subsection{Precise Call/Return Match}

show the precision of stack

\subsection{Termination}

show that number of continuation is finite.

\subsection{Soundness}

\subsection{Complexity}

\section{Detour: Shiver's \textit{k}-CFA}

\section{Perspective}

\subsection{Defunctionalization}

\subsection{Refunctionalization}

\section{Related Work}

\section{Conclusion}

%% Acknowledgments
\begin{acks}                            %% acks environment is optional
                                        %% contents suppressed with 'anonymous'
  %% Commands \grantsponsor{<sponsorID>}{<name>}{<url>} and
  %% \grantnum[<url>]{<sponsorID>}{<number>} should be used to
  %% acknowledge financial support and will be used by metadata
  %% extraction tools.
  This material is based upon work supported by the
  \grantsponsor{GS100000001}{National Science
    Foundation}{http://dx.doi.org/10.13039/100000001} under Grant
  No.~\grantnum{GS100000001}{nnnnnnn} and Grant
  No.~\grantnum{GS100000001}{mmmmmmm}.  Any opinions, findings, and
  conclusions or recommendations expressed in this material are those
  of the author and do not necessarily reflect the views of the
  National Science Foundation.
\end{acks}

%% Bibliography
\bibliography{references}

%% Appendix
\appendix
\section{Appendix}

Text of appendix \ldots

\end{document}
