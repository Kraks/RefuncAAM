%% For double-blind review submission, w/o CCS and ACM Reference (max submission space)
\documentclass[acmsmall,review,anonymous]{acmart}\settopmatter{printfolios=true,printccs=false,printacmref=false}
%% For double-blind review submission, w/ CCS and ACM Reference
%\documentclass[acmsmall,review,anonymous]{acmart}\settopmatter{printfolios=true}
%% For single-blind review submission, w/o CCS and ACM Reference (max submission space)
%\documentclass[acmsmall,review]{acmart}\settopmatter{printfolios=true,printccs=false,printacmref=false}
%% For single-blind review submission, w/ CCS and ACM Reference
%\documentclass[acmsmall,review]{acmart}\settopmatter{printfolios=true}
%% For final camera-ready submission, w/ required CCS and ACM Reference
%\documentclass[acmsmall]{acmart}\settopmatter{}


%% Journal information
%% Supplied to authors by publisher for camera-ready submission;
%% use defaults for review submission.
\acmJournal{PACMPL}
\acmVolume{1}
\acmNumber{ICFP} % CONF = POPL or ICFP or OOPSLA
\acmArticle{1}
\acmYear{2018}
\acmMonth{1}
\acmDOI{} % \acmDOI{10.1145/nnnnnnn.nnnnnnn}
\startPage{1}

%% Copyright information
%% Supplied to authors (based on authors' rights management selection;
%% see authors.acm.org) by publisher for camera-ready submission;
%% use 'none' for review submission.
\setcopyright{none}
%\setcopyright{acmcopyright}
%\setcopyright{acmlicensed}
%\setcopyright{rightsretained}
%\copyrightyear{2018}           %% If different from \acmYear

%% Bibliography style
\bibliographystyle{ACM-Reference-Format}
%% Citation style
%% Note: author/year citations are required for papers published as an
%% issue of PACMPL.
\citestyle{acmauthoryear}   %% For author/year citations


%%%%%%%%%%%%%%%%%%%%%%%%%%%%%%%%%%%%%%%%%%%%%%%%%%%%%%%%%%%%%%%%%%%%%%
%% Note: Authors migrating a paper from PACMPL format to traditional
%% SIGPLAN proceedings format must update the '\documentclass' and
%% topmatter commands above; see 'acmart-sigplanproc-template.tex'.
%%%%%%%%%%%%%%%%%%%%%%%%%%%%%%%%%%%%%%%%%%%%%%%%%%%%%%%%%%%%%%%%%%%%%%


%% Some recommended packages.
\usepackage{booktabs}   %% For formal tables:
                        %% http://ctan.org/pkg/booktabs
\usepackage{subcaption} %% For complex figures with subfigures/subcaptions
%% http://ctan.org/pkg/subcaption
\usepackage{syntax}

\begin{document}

%% Title information
\title[Short Title]{Pushdown Control-Flow Analysis via Refunctionalization}         %% [Short Title] is optional;
                                        %% when present, will be used in
                                        %% header instead of Full Title.
\titlenote{with title note}             %% \titlenote is optional;
                                        %% can be repeated if necessary;
                                        %% contents suppressed with 'anonymous'
\subtitle{Filling the Gap between Abstracting Abstract Machine and Abstracting Definitional Interpreter}        %% \subtitle is optional
\subtitlenote{with subtitle note}       %% \subtitlenote is optional;
                                        %% can be repeated if necessary;
                                        %% contents suppressed with 'anonymous'


%% Author information
%% Contents and number of authors suppressed with 'anonymous'.
%% Each author should be introduced by \author, followed by
%% \authornote (optional), \orcid (optional), \affiliation, and
%% \email.
%% An author may have multiple affiliations and/or emails; repeat the
%% appropriate command.
%% Many elements are not rendered, but should be provided for metadata
%% extraction tools.

%% Author with single affiliation.
\author{First1 Last1}
\authornote{with author1 note}          %% \authornote is optional;
                                        %% can be repeated if necessary
\orcid{nnnn-nnnn-nnnn-nnnn}             %% \orcid is optional
\affiliation{
  \position{Position1}
  \department{Department1}              %% \department is recommended
  \institution{Institution1}            %% \institution is required
  \streetaddress{Street1 Address1}
  \city{City1}
  \state{State1}
  \postcode{Post-Code1}
  \country{Country1}                    %% \country is recommended
}
\email{first1.last1@inst1.edu}          %% \email is recommended

%% Author with two affiliations and emails.
\author{First2 Last2}
\authornote{with author2 note}          %% \authornote is optional;
                                        %% can be repeated if necessary
\orcid{nnnn-nnnn-nnnn-nnnn}             %% \orcid is optional
\affiliation{
  \position{Position2a}
  \department{Department2a}             %% \department is recommended
  \institution{Institution2a}           %% \institution is required
  \streetaddress{Street2a Address2a}
  \city{City2a}
  \state{State2a}
  \postcode{Post-Code2a}
  \country{Country2a}                   %% \country is recommended
}
\email{first2.last2@inst2a.com}         %% \email is recommended
\affiliation{
  \position{Position2b}
  \department{Department2b}             %% \department is recommended
  \institution{Institution2b}           %% \institution is required
  \streetaddress{Street3b Address2b}
  \city{City2b}
  \state{State2b}
  \postcode{Post-Code2b}
  \country{Country2b}                   %% \country is recommended
}
\email{first2.last2@inst2b.org}         %% \email is recommended


%% Abstract
%% Note: \begin{abstract}...\end{abstract} environment must come
%% before \maketitle command
\begin{abstract}
  Abstracting abstract machines (AAM) is a systematic methodology 
  for constructing abstract interpreter that derived from a concrete
  small-step abstract machine. Recent progress applies the same idea 
  on concrete definitional interpreter and obtains a big-step 
  abstracting definitional interpreter (ADI) written in monadic style.
  Danvy shows the correspondence between abstract machines and interpreters.
  In this paper, we study the relations between small-step AAM and 
  big-step ADI.

  We show their correspondence and how to syntactically transform a small-step 
  abstracting abstract machine to a big-step abstracting definitional 
  interpreter.
  The transformations include fusing, disentangling, refunctionalizing 
  and monadification(un-CPS), and all of them properly handle the 
  collecting semantics and the non-determinism of abstract interpretation. 
  After each transformation, we also obtain an intermediary form of 
  abstract interpreter.
  Remarkably, follow the idea that evaluation contexts are defuntionalized 
  continuations, we reveal how precise call/return match in control-flow 
  analysis is obtained by refunctionalizing a small-step abstracting 
  abstract machine,  % with unbounded stack
  as well as explain how it been lost by defunctionalizing the higher-order
  continuations into an algebraic data type. Further, we show how to turn 
  precise call/return match into computable analysis.
  
  TODO: Maybe something Shiver's formulation of CFA in denotational style. 
 
\end{abstract}

%% 2012 ACM Computing Classification System (CSS) concepts
%% Generate at 'http://dl.acm.org/ccs/ccs.cfm'.
\begin{CCSXML}
<ccs2012>
<concept>
<concept_id>10011007.10011006.10011008</concept_id>
<concept_desc>Software and its engineering~General programming languages</concept_desc>
<concept_significance>500</concept_significance>
</concept>
<concept>
<concept_id>10003456.10003457.10003521.10003525</concept_id>
<concept_desc>Social and professional topics~History of programming languages</concept_desc>
<concept_significance>300</concept_significance>
</concept>
</ccs2012>
\end{CCSXML}

\ccsdesc[500]{Software and its engineering~General programming languages}
\ccsdesc[300]{Social and professional topics~History of programming languages}
%% End of generated code


%% Keywords
%% comma separated list
\keywords{pushdown analysis, abstract interpretation, static analysis}  %% \keywords are mandatory in final camera-ready submission


%% \maketitle
%% Note: \maketitle command must come after title commands, author
%% commands, abstract environment, Computing Classification System
%% environment and commands, and keywords command.
\maketitle

\section{Introduction}

\subsection{Motivation}

\subsection{Contributions}

the essence of abstracting defintional interpreter.

\begin{itemize}
\item The main contribution of this paper is we fill the gap
  between small-step AAM and big-step ADI by developing a series of
  systematic transformations. The correspondence between abstract
  machines and interpreters exists not only in concrete semantics artifact,
  but also in abstract semantics artifact.
  Galois connection?

\item We present an abstract interpreter obtained from refunctionalizing
  an small-step abstracting abstract machine. 
  This abstract interpreter naturally has pushdown control-flow property
  since we represent the evluation context of analyzed language using
  evluation context of the meta-language.
  We also analyze its termination, soundness and complexity.

\item Further, we show that defunctionalization and refunctionalization
  of abstract interpreters play important roles for call stack of analyzed language.
  Given this insight, we reveal that why small-step AAM has no pushdown
  control-flow property (if without extra effort), as well as explain
  that why ADI naturally extends pushdown control-flow property from its
  defining language.

\end{itemize}

\subsection{Outline}

\subsection{Style}

We use Scala language to demonstrate the idea and each step of transformations.

\section{Background}

In this section, we frist describe the target language we will
analyze and a concrete CESK machine as operational semantics for it.
Then we review some recent works in control-flow analysis for
higher-order languages:
1) abstracting abstract machines (AAM) as an abstract interpretation
counterpart of CESK machines, and
2) the precise call/return match problem in control-flow analysis and its existing solutions.
We will show in section 3 that the abstracting abstract machines form the basis of our transformations,
and call/return match is obtained during transformation.

\subsection{A-Normal Form $\lambda$-Calculus} \label{anfsyntax}

Tranditionally, continuation-passing style (CPS) is a popular intermediate representation
for analyzing functional programs because of it exposes control transfer explicitly
and simplifies the analysis \cite{Shivers:1991:SSC:115865.115884, Shivers:1988:CFA:53990.54007}.
Here, we choose to use a direct-style intermediate representation as target for
clarity but without losing simplicty and generality.
The transformations we will show in the rest of this paper
also work on abstract machines for plain $\lambda$-calculus languages.

To begin, we now present the concrete syntax of a call-by-value $\lambda$-calculus language in
administrative normal form (ANF) \cite{flanagan1993essence}.

\begin{verbatim}
e \in Exp ::= ae
            | (let ([x (f ae)]) e)
            | (letrec ([x ae] ...) e) (TODO: maybe not include letrec?)
ae \in AExp ::= x
              | lam
lam \in Lam ::= (lambda (x) e)
x ... are variable names
\end{verbatim}

In ANF, all the function applications must be administrated in a \texttt{let} expression,
and then bounded to a variable name under current environment.
We can use \texttt{letrec} to create mutually recursive function bindings. But \texttt{letrec}
only accepts atomic expressions that appear at the right-hand side of a binding.
The atomic expressions $ae$ are either a variable, or a literal \texttt{lambda} term, which
both can be evaluated in a single step.

The abstract syntax can be straightforwardly described in Scala as follows:

\begin{verbatim}
sealed trait Expr
case class Binding(x: String, e: Expr)

case class Var(x: String) extends Expr
case class App(e1: Expr, e2: Expr) extends Expr
case class Lam(x: String, body: Expr) extends Expr
case class Let(x: String, e: App, body: Expr) extends Expr
case class Letrec(bds: List[Binding], body: Expr) extends Expr
\end{verbatim}

\subsection{CESK Machine}

\subsubsection{Machine Components}

\textit{C}ontrol \textit{E}nvionment \textit{S}tore \textit{K}ontinuation (CESK) machine is an
abstract machine for describing semantics of and evaluating $\lambda$-calculus. (TODO: cite)
CESK machine models program execution as state transitions in small-step fashion. As its name suggested,
a machine state has four components: 1) Control is the expression currently being evaluated.
2) Environment is the a map that contains the address of a variable that in the lexical
scope of control.
3) Store models the heap of a program, and it is also a map but maps from addresses to values.
The addresses are infinite numbers starting from 0.
In our toy language, the only catagory of value is closure, i.e., a function paired with an environment.
4) Continuation represents the program stack.

We show the Scala representations for the components of CESK machine.

\begin{verbatim}
type Addr = Int
type Env = Map[String, Addr]
type Store = Map[Addr, Storable]

abstract class Storable
case class Clos(v: Lam, env: Env) extends Storable

case class Frame(x: String, e: Expr, env: Env)
type Kont = List[Frame]

case class State(e: Expr, env: Env, store: Store, k: Kont)
\end{verbatim}

It is worth noting that the continuation class \texttt{Kont} is defined as a list of frames.
The frame class \texttt{Frame} stores the information of call-site, i.e., the information that
can be used to resume the interrupted computation.
A \texttt{Frame} constitutes of a variable name \texttt{x} to be bind later, a control expression
that resumes to, and its environment.

\subsubsection{Single-step Evaluation}
Before go into describing how the machine evaluates expressions, we firstly define several helper functions.
As we mentioned in section \ref{anfsyntax}, the atomic expressions are either a variable, or
a literal \texttt{lambda} term. So the helper function \texttt{atomicEval} handles these two
cases and evaluates atomic expressions to closures in a straightforward way.
The \texttt{alloc} function generates a fresh address, and always allocates an unique integer under the domain
of store.
The \texttt{isAtomic} function is used as a predicate of whether the expression is atomic.

\begin{verbatim}
def atomicEval(e: Expr, env: Env, store: Store): Storable = e match {
  case Var(x) => store(env(x))
  case lam@Lam(x, body) => Clos(lam, env)
}
def alloc(store: Store): Addr = store.keys.size + 1
def isAtomic(e: Expr): Boolean = e.isInstanceOf[Var] || e.isInstanceOf[Lam]
\end{verbatim}

Now, we can faithfully describe the state transition function \texttt{step},
that given a machine state, function \texttt{step} determines its successor state.

\begin{verbatim}
def step(s: State): State = s match {
  case State(Let(x, App(f, ae), e), env, store, k) if isAtomic(ae) =>
    val Clos(Lam(v, body), env_c) = atomicEval(f, env, store)
    val addr = alloc(store)
    val new_env = env_c + (v -> addr)
    val new_store = store + (addr -> atomicEval(ae, env, store))
    val frame = Frame(x, e, env)
    State(body, new_env, new_store, frame::k)
  case State(ae, env, store, k) if isAtomic(ae) =>
    val Frame(x, e, env_k)::ks = k
    val addr = alloc(store)
    val new_env = env_k + (x -> addr)
    val new_store = store + (addr -> atomicEval(ae, env, store))
    State(e, new_env, new_store, ks)
}
\end{verbatim}

The first case is that the control of current state is a \texttt{Let} expression,
and its right-hand side is an function application.
By calling function \texttt{atomicEval}, we obtain the closure that callee \texttt{f} stands for.
The successor state's control then transfers to the \texttt{body} expression of closure,
with an updated environment and an update store. The new environment is extended
from closure's environment and mapped \texttt{v} to a fresh address \texttt{addr}.
The new store also extends with \texttt{addr} mapping to the value of \texttt{ae},
which is evaluated from \texttt{atomicEval}.
In the meanwhile, a new frame \texttt{frame} is pushed onto the stack \texttt{k}.
This \texttt{frame} contains the left-hand side variable name of \texttt{Let},
the body expression, and the lexical environment of body expression.

The second case for \texttt{step} is the control is transferred to an atomic expression.
We extract the top frame of continuations firstly.
The control (that is an atomic expression) of current state is the evaluated term
that being bind to the variable \texttt{x} from the top frame.
The environment and store are updated with \texttt{x} mapping to the closure value of \texttt{ae}.
Then the successor state is transferred to expression \texttt{e} from the top frame,
which is the body of a \texttt{Let} expression, with the updated environment, store, and
the rest of stack \texttt{ks}.

\subsubsection{Valuation}

To run the program, we use function \texttt{inject} to construct an initial machine
state given an expression. Then the \texttt{drive} function is used to evaluate
to a final state by iteratively apply \texttt{step} on current state until we reach a state
that the control is an atomic expression and the continuation stack is empty.
And of course, we can extract the value from the final state at last.

\begin{verbatim}
def drive(s: State): State = s match {
  case State(ae, _, _, Nil) if isAtomic(ae) => s
  case _ => drive(step(s))
}

def inject(e: Expr): State = State(e, Map(), Map(), Nil)
def eval(e: Expr): State = drive(inject(e))
\end{verbatim}

\subsection{Abstracting Abstract Machine} \label{aam}

store allocated continuation
binding store, continuation store
store maps from addresses to a set of values
store update is a join
the step function is non-deterministic

\begin{verbatim}
case class Store[K,V](map: Map[K, Set[V]]) {
  def apply(addr: K): List[V] = map(addr).toList
  def update(addr: K, d: Set[V]): Store[K,V] = {
    val oldd = map.getOrElse(addr, Set())
    Store[K, V](map ++ Map(addr -> (d ++ oldd)))
  }
  def update(addr: K, sd: V): Store[K,V] = update(addr, Set(sd))
}

type Time = List[Expr]

case class BAddr(x: String, time: Time)
type Env = Map[String, BAddr]

abstract class Storable 
case class Clos(v: Lam, env: Env) extends Storable
type BStore = Store[BAddr, Storable]

abstract class KAddr
case object Halt extends KAddr
case class ContAddr(tgt: Expr, time: Time) extends KAddr

case class Frame(x: String, e: Expr, env: Env)
case class Cont(frame: Frame, kaddr: KAddr)
type KStore = Store[KAddr, Cont]

case class State(e: Expr, env: Env, bstore: BStore, 
                 kstore: KStore, k: KAddr, time: Time)
\end{verbatim}

\begin{verbatim}
def allocBind(x: String, time: Time): BAddr = BAddr(x, time)
def allocKont(tgtExpr: Expr, tgtEnv: Env, tgtStore: BStore, time: Time): KAddr = 
  ContAddr(tgtExpr, time)
def k: Int = 0
def tick(s: State): Time = (s.e::s.time).take(k)

def inject(e: Expr): State = 
  State(e, Map(), 
        Store[BAddr, Storable](Map()), 
        Store[KAddr, Cont](Map(Halt -> Set())), 
        Halt, List())

def atomicEval(e: Expr, env: Env, bstore: BStore): Set[Storable] = e match {
  case Var(x) => bstore(env(x)).toSet
  case lam@Lam(x, body) => Set(Clos(lam, env))
}
\end{verbatim}

\begin{verbatim}
def step(s: State): List[State] = {
  val newTime = tick(s)
  s match {
    case State(Let(x, App(f, ae), e), env, bstore, kstore, kaddr, time) if isAtomic(ae)=>
      for (Clos(Lam(v, body), c_env) <- atomicEval(f, env, bstore).toList) yield {
        val baddr = allocBind(v, newTime)
        val newEnv = c_env + (v -> baddr)
        val newBStore = bstore.update(baddr, atomicEval(ae, env, bstore))
        val newKAddr = allocKont(body, c_env, newBStore, newTime)                                                          
        val newKStore = kstore.update(newKAddr, Cont(Frame(x, e, env), kaddr))
        State(body, newEnv, newBStore, newKStore, newKAddr, newTime)
      }
    case State(ae, env, bstore, kstore, kaddr, time) if isAtomic(ae)=>
      for (Cont(Frame(x, e, f_env), f_kaddr) <- kstore(kaddr).toList) yield {
        val baddr = allocBind(x, newTime)
        val newEnv = f_env + (x -> baddr)
        val newStore = bstore.update(baddr, atomicEval(ae, env, bstore))
        State(e, newEnv, newStore, kstore, f_kaddr, newTime)
      }
  }
}
\end{verbatim}

Comparing with concrete CESK machine, the
\texttt{drive} function performs a collecting semantics rather than valuation
semantics. That is, for the purpose of analyzing a program, \texttt{drive}
collects all the intermediate machine states as program is abstractly executing.
This is a variant of work list algorithm that always applies \texttt{step}
function to the first element of \texttt{todo} and adds it to \texttt{seen}
if there is an unexplored one. If the work list is empty, the function just
returns the set of explored states up to now.

\begin{verbatim}
def drive(todo: List[State], seen: Set[State]): Set[State] = todo match {
  case Nil => seen
  case hd::tl if seen.contains(hd) => drive(tl, seen)
  case hd::tl => drive(step(hd).toList ++ tl, seen + hd)
}

def analyze(e: Expr): Set[State] = drive(List(inject(e)), Set())
\end{verbatim}

%\subsection{Monadic Abstract Interpreter}
%Maybe not here?

\subsection{Call/Return Match in Control-Flow Analysis}

describe problem, show some examples, why it is important

In this section, we firstly describe a naive and uncomputable solution to the
problem; then briefly review several recent research works that enable
computable and precise call/return match. Even though the first solution is
uncomputable, but as we will see in section 3, it is the starting point of our
transformation. 

\subsubsection{AAM with Unbound Stack}

uncomputable solutions

\begin{verbatim}
case class State(e: Expr, env: Env, bstore: BStore, konts: List[Frame], time: Time)

def step(s: State): List[State] = {
  val newTime = tick(s)
  s match {
    case State(Let(x, App(f, ae), e), env, bstore, konts, time) =>
      for (Clos(Lam(v, body), c_env) <- atomicEval(f, env, bstore).toList) yield {
        val frame = Frame(x, e, env)
        val baddr = allocBind(v, newTime)
        val newEnv = c_env + (v -> baddr)
        val newStore = bstore.update(baddr, atomicEval(ae, env, bstore))
        State(body, newEnv, newStore, frame::konts, newTime)
      }
    case State(ae, env, bstore, konts, time) if isAtomic(ae) =>
      konts match {
        case Nil => List()
        case Frame(x, e, f_env)::konts =>
          val baddr = allocBind(x, newTime)
          val newEnv = f_env + (x -> baddr)
          val newStore = bstore.update(baddr, atomicEval(ae, env, bstore))
          List(State(e, newEnv, newStore, konts, newTime))
      }
  }
}

\end{verbatim}

\subsubsection{Computable Solutions}

Recent years, there are great efforts \cite{vardoulakis2010cfa2,
  earl2012introspective, gilray2016pushdown}
to achieve precise call/return match based
on small-step abstracting abstract machines which is we described in section \ref{aam}.

go over existing computable solutions P4F ANF, not works on direct style,PDCFA,AAC

\section{FROM AAM TO ADI}

We start from the AAM with unbounded stack.

\subsection{Fusing}

\subsection{Disentangling}

\subsection{Refunctionalization}

allocating continuations of defined language on the heap of defining language.


%probably Store Widening \& Abstract Garbage Collection
%monadic
\subsection{Monadification: From Refunctionalized AAM to ADI}

For one direction, we can transform the refunctionalized AAM to abstracting definitional interpreter
by adding monads.

\subsection{Un-CPS: From Refunctionalized AAM to Direct Style}

In another direction, we can transform the refunctionalized AAM to direct style, but explictly use 
side effects such as assignment and mutation to handle collecting semantics and non-determinism.

\section{Refunctionalized AAM}

\subsection{Precise Call/Return Match}

show the precision of stack

\subsection{Termination}

show that number of continuation is finite.

\subsection{Soundness}

\subsection{Complexity}

\section{Perspective}

\subsection{Defunctionalization}

specify higher-order definitional interpreters using first-order means (cite)

\subsection{Refunctionalization}

\section{Detour: Shiver's \textit{k}-CFA}

\section{Related Work}

\section{Conclusion}

%% Acknowledgments
\begin{acks}                            %% acks environment is optional
                                        %% contents suppressed with 'anonymous'
  %% Commands \grantsponsor{<sponsorID>}{<name>}{<url>} and
  %% \grantnum[<url>]{<sponsorID>}{<number>} should be used to
  %% acknowledge financial support and will be used by metadata
  %% extraction tools.
  This material is based upon work supported by the
  \grantsponsor{GS100000001}{National Science
    Foundation}{http://dx.doi.org/10.13039/100000001} under Grant
  No.~\grantnum{GS100000001}{nnnnnnn} and Grant
  No.~\grantnum{GS100000001}{mmmmmmm}.  Any opinions, findings, and
  conclusions or recommendations expressed in this material are those
  of the author and do not necessarily reflect the views of the
  National Science Foundation.
\end{acks}

%% Bibliography
\bibliography{references}

%% Appendix
\appendix
\section{Appendix}

Text of appendix \ldots

\end{document}
